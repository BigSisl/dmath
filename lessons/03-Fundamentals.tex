\section{Fundamentals}

\subsection{Wachstum von Funktionen}
$ f $= ''\textit{sehr komplizierte Funktion}`` \\
$ g $= ''\textit{einfachere Funktion}`` \\
$ |f(x)| \leq C|g(x)|, \forall x > k $ \\
$ f(x) = \mathcal{O}(g(x)) $

\subsection{Exponentialfunktionen}
\begin{tabular}{ll}
    $ a^r * a^s = a^{r+s} $ \\
    $ \frac{a^r}{a^s} = a^{r-s} $ \\
    $ (a^r)^s = (a^s)^r = a^{r*s} $ \\
\end{tabular}

\subsection{Logarithmusfunktionen}
\begin{tabular}{ll}
    $ log_a(u * v) = log_a(u) + log_a(v) $ \\
    $ log_a(\frac{u}{v}) = log_a(u) - log_a(v) $ \\
    $ log_a(u^v) = v * log_a(u) $ \\
\end{tabular}

\subsection{Komplexität von Algorithmen}
\begin{tabular}{ll}
    konstant & $ \mathcal{O}(1) $ \\
    logarithmisch & $ \mathcal{O}(log n) $ \\ 
    linear & $ \mathcal{O}(n) $ \\
    n log n & $ \mathcal{O}(n * log n) $ \\
    polynomial & $ \mathcal{O}(n^b) $ \\
    exponentiell & $ \mathcal{O}(b^n), b > 1$ \\
    faktorielle & $ \mathcal{O}(n!) $ \\
\end{tabular}

\subsection{Zahlen und Division}
\begin{tabular}{l}
    $ a|b \land a|c \rightarrow a|(b + c) $ \\
    $ a|b \rightarrow \forall c(a|bc) $ \\
    $ a|b \land b|c \rightarrow a|c $ \\
\end{tabular}

\subsection{Primzahl}
$ \not \exists a (a|n \land (1 < a < n)) $

\subsection{Mersenne Primes}
$ M_n = 2^p - 1, p \in $ ''\textit{Primzahlen}``

\subsection{Primzahlsatz}
$ \pi (x) \approx \frac{x}{ln(x)} $

\subsection{ggT und kgV}
\begin{tabular}{l}
    $ a = dq + r, $ wobei $ (0 \leq r < d)$ \\
    $ q = a $ div $ d $ und $ r = a $ mod $ d $ \\
    $ ab = ggT(a, b) * kgV(a, b) $ \\
\end{tabular}

\subsection{Kongruenz}
$ a \equiv b $ mod $ m $, $ m|(a - b) $ 

\subsection{Addition zweier Matrizen}
$ A + B = 
\begin{bmatrix}
    a_{11} + b_{11} & a_{12} + b_{12} & \cdots & a_{1n} + b_{1n} \\
    a_{21} + b_{21} & a_{22} + b_{22} & \cdots & a_{2n} + b_{2n} \\
    \vdots  & \vdots  & \ddots & \vdots  \\
    a_{m1} + b_{m1} & a_{m2} + b_{m2} & \cdots & a_{mn} + b_{mn} 
\end{bmatrix}$

\subsection{Multiplikation einer Matrix mit einer Zahl}
$ \alpha A = 
\begin{bmatrix}
    \alpha a_{11} & \alpha a_{12} & \cdots & \alpha a_{1n} \\
    \alpha a_{21} & \alpha a_{22} & \cdots & \alpha a_{2n} \\
    \vdots  & \vdots  & \ddots & \vdots  \\
    \alpha a_{m1} & \alpha a_{m2} & \cdots & \alpha a_{mn} 
\end{bmatrix}$

\subsection{Multiplikation von Matrizen}
\begin{tabular}{cc}
    $A \times B = C $ &
    $\begin{bmatrix}
        b_{11} & \cdots & b_{1n} \\
        \vdots & \ddots & \vdots  \\
        b_{m1} & \cdots & b_{mn} 
    \end{bmatrix}$ \\
    & \\
    $\begin{bmatrix}
        a_{11} & \cdots & a_{1n} \\
        \vdots & \ddots & \vdots  \\
        a_{m1} & \cdots & a_{mn} 
    \end{bmatrix}$ & 
    $\begin{bmatrix}
        c_{11} & \cdots & c_{1n} \\
        \vdots & \ddots & \vdots  \\
        c_{m1} & \cdots & c_{mn} 
    \end{bmatrix}$ \\
\end{tabular} \\ \\

$c_{11} = (a_{11} * b_{11}) + (a_{12} * b_{21}) + \dots + (a_{1n} * b_{m1}) $

\subsection{Transponierte Matrix}
$A^T$ durch Vertauschen von Zeilen und Spalten

\subsection{Symmetrie einer Matrix}
ist symmetrisch, falls $A^T = A$ \\
ist antisymmetrisch, falls $A^T = -A$

\subsection{Einheitsmatrix}
$I_n$ ist eine Matrix bei der alle Elemente auf der Diagonalen Eins und alle anderen Null sind

\subsection{Inverse Matrix}
$A^{-1} * A = A * A^{-1} = I_n$

\subsection{Boolsches Produkt zweier Matrizen}
$ A \odot B = [c_{ij}] $, \\
wobei $c_{ij} = (a_{i1} \land b_{1j}) \lor (a_{i2} \land b_{2j}) \lor \dots \lor (a_{in} \land b_{nj}) $

