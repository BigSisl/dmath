\subsection{Restklassen}
$ [r] = \{x \in Z | x \equiv r \mod n\} $

\subsection{Rechenregeln für das modulare Rechnen}
\begin{tabular}{l}
    $a \oplus_n b = b \oplus_n a = a + b \mod n = R_n(a + b) $ \\
    $a \odot_n b = b \odot_n a = a * b \mod n = R_n(a * b)$ \\
    $a \odot_n (b \oplus_n c) = (a \odot_n b) \oplus_n (a \odot_n c)$ \\
\end{tabular} \newline

\subsection{Potenzieren modulo n}
$ x^m = x^{2 * k + l} = x^{2 * k} * x^l = (x^k)^2 * x^l $

\subsection{Square and Multiply Algorithm}
\begin{tabular}{rl}
    1. & Exponent binär schreiben \\
    2. & Q bedeutet quadrieren und M multiplizieren \\
    3. & Ersetze 1 durch QM und 0 durch Q \\
    4. & das erste (links) QM streichen \\
    5. & Reihenfolge von Quadrieren und Multipliziere \\
    6. & Exponent einsetzten \\
    7. & entsprechend Quadrieren und Multiplizieren \\
    8. & immer wieder modular reduzieren \\
\end{tabular}

\subsection{Nullteiler}
$ a \in \mathbb{Z}_n, a \neq 0, b \in \mathbb{Z}_n, b \neq 0 $ \\
falls $ a \odot_n b = 0 $, dann ist $a$ Nullteiler von $\mathbb{Z}_n$

\subsection{Inverse Elemente}
$ \mathbb{Z}_n^* = \{ a \in \mathbb{Z}_n | ggT(a, n) = 1 \} $ \\
$ a^{-1} = R_p(a^{p-2}) = a^{p-2} \mod p $, ($p$ = Primzahl)

\subsection{Primitive Elemente / Erzeugende}
falls jedes Element $a \in \mathbb{Z}_p^* $ eine Potent von $z$ ist

\subsection{Einwegfunktionen}
\begin{tabular}{ll}
    Quadrieren modulo n & $ x \mapsto x^2 \mod n$ \\
    Potenzieren modulo n & $ x \mapsto x^e \mod n$ \\
    Exponentialfunktion modulo p & $ x \mapsto b^x \mod p$ \\
\end{tabular} \\

$ n = pq $ \textit{(Multiplikation zweier Primzahlen)}

\subsection{Modulare Quadratwurzeln}
$ \sqrt{a} \mod n = \{ x \in \mathbb{Z}_n^* | x^2 = a \mod n \} $ \\
=> Für ein $a$ kann es mehrere Quadratwurzeln geben

\subsection{diskrete Logarithmus}
exp$_b(k) = b^k \mod p$

\subsection{Diffie-Hellmann Schlüsselvereinbarung}
\begin{tabular}{ll}
    1. & Wähle zwei natürliche Zahlen $p$ und $s$ \\
    2. & $A$ wählt eine Zufallszahl $a < p$ \\
    & $A$ berechnet $\alpha = s^a \mod p$ \\
    & $A$ sendet $\alpha$ über einen Kanal an $B$ \\
    3. & $B$ wählt eine Zufallszahl $b < p$ \\
    & $B$ berechnet $\beta = s^b \mod p$ \\
    & $B$ sendet $\beta$ über einen Kanal an $A$ \\
    4. & $A$ berechnet $\beta ^a \mod p = s^{b*a} \mod p$ \\
    5. & $B$ berechnet $\alpha ^b \mod p = s^{b*a} \mod p$ \\
    6. & Beide haben den gemeinsamen Schlüssel  \\
\end{tabular} \\
