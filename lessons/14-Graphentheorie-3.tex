\subsection{Gewichtete Graphen}
$ w : E \rightarrow ( 0, \infty) $ \\

\textit{Länge / Gewicht eines Weges:} \\
$w(u_0, u_1, \dots, u_n) = w(u_0, u_1) + \dots + w(u_{n-1}, u_n) $ \\

\textit{Abstand $d(u,v)$:} \\
Minimum der Längen aller Wege von $u$ nach $v$

\subsection{Minimale aufspannende Bäume}
Bedingung: $T \subseteq E $ \\
\\
$ w(T) = \sum_{\{u,v\} \in T} w(u,v) $

\subsection{Algorithmus von Dijkstra}
Länge des kürzeseten Weges von $a$ nach $u$ für jeden Knoten $u$ aus $V$ \\

\begin{tabular}{@{}lll}
    $ S := \emptyset $ & $ L(a) := 0$ & $L(u) := \infty $ \\
\end{tabular}\\

\textit{Wiederhole:}\\
\begin{tabular}{@{}ll}
    1. & Wähle einen Knoten $s \in V - S$ mit minimalem $L(s)$ \\
    2. & Falls $L(s) = \infty$, dann HALT \\
    3. & Füge der Menge $S$ den Knoten $s$ hinzu \\
    4. & Falls $S = V$, dann HALT \\
    5. & Für jeden Nachbarn $y \in V - S$ des Knoten $s$: \\
    & Falls $L(y) > L(s) + w(s, y)$, ersetze $L(y)$ durch \\ 
    & $L(s) + w(s, y)$; andernfalls tue nichts \\
\end{tabular}        

\subsection{Algorithmus von Prim}
minimalen aufspannenden Baum $T$ von $G$ \\

\begin{tabular}{@{}ll}
    $S := \{ a \} $ & $ T := \emptyset $ \\
\end{tabular}\\

\textit{Wiederhole so lange wie möglich:}\\
\begin{tabular}{@{}ll}
    1. & Wähle eine Kante $\{x, y\} \in E$ minimalen Gewichts \\ 
    & mit $x \in S$ und $y \in V - S$ \\
    2. & Füge der Menge $S$ den Knoten $y$ hinzu \\
    3. & Füge der Menge $T$ die Kante $\{x, y\}$ hinzu \\
\end{tabular}

\subsection{Algorithmus von Kruskal}
berechnet einen minimalen aufspannenden
Wald $T$ und die Menge $R$ der Zusammenhangskomponenten

\begin{tabular}{@{}ll}
    $R := \{ \{x\} | x \in V \} $ & $ T := \emptyset $ \\
\end{tabular}\\

\textit{Wiederhole so lange wie möglich:}\\
\begin{tabular}{@{}ll}
    1. & Wähle eine Kante $\{x, y\} \in E - T$ minimalen \\ 
    & Gewichts, so dass $x$ und $y$ nicht zur gleichen \\
    & Klasse von $R$ gehören \\
    2. & Ersetze in $R$ die beiden Klassen von $x$ und $y$ \\
    & durch ihre Vereinigung.\\
    3. & Füge der Menge $T$ die Kante $\{x, y\}$ hinzu\\
\end{tabular}

