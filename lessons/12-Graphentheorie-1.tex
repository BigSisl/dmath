\section{Graphentheorie 1}

\subsection{Grade}
\begin{tabular}{@{}ll}
    Eckengrad & $ \sum_{v \in V} \deg(v) = 2 * |E| $ \\
    Maximalgrad & $ \Delta(G) = \max_{v \in V(G)} {\deg}(v) $ \\
    Maximalgrad & $ \delta(G) = \min_{v \in V(G)} {\deg}(v) $ \\
    & \\
    \textit{Spezielle Ecken:} & \\
    isolierte Ecke & $ \deg(v) = 0 $ \\
    Endecke & $ \deg(v) = 1 $ \\
\end{tabular}

\subsection{Isomorphe Graphen}
isomorph, falls es eine Bijektion $ f : V \rightarrow V'$ \\
$ \{ u, v \} \in E \Leftrightarrow \{f(u), f(v)\} \in E' $

\subsection{Vollständiger Graph}
Vollständiger Graph mit $n$ Knoten: genau eine Kante zwischen je zwei Knoten ($m$ Kanten). \\
\\
$ m = \binom{n}{2} = \frac{(n-1)n}{2} $ 

\subsection{Eigenschaften eines Baumes}
\begin{tabular}{@{}ll}
    Baum mit $n$ Knoten & $n-1$ Kanten \\
    Baum mit $i$ inneren Knoten & $n=m \cdot i + 1$ Knoten \\
    $m$-facher Baum der Höhe $h$ & höchstens $m^h$ Blätter \\
\end{tabular}

\subsection{Vollständige bipartite Graphen}
Bedingung: $ U \cup W = V $ und $ U \cap W = \emptyset $ \\
\\
1. keine Kante zwischen Knoten aus $U$ \\
2. keine Kante zwischen Knoten aus $W$ \\
3. Knoten aus sind genau durch eine Kante verbunden \\
4. $\forall u \in U$, $u$ ist mit jedem Knoten aus $W$ verbunden \\
5. $\forall w \in W$, $w$ ist mit jedem Knoten aus $U$ verbunden

\subsection{Page-Rank-Algorithmus}
Gewicht der Seite $PR_i$ in einem Netz mit $N$ Seiten \\
Dämpfungsfaktor $d$ mit $0 \leq d \leq 1 $ \\
$C_j$ von Seite $j$ abgehende Links \\
\\
$PR_i = \frac{1-d}{N} + d \cdot \sum_{j} {{PR_j}\over{C_j}}$

\subsection{Matrizen}
\begin{tabular}{@{}lll}
    $n$ Ecken, $m$ Kanten \\
    \\
    \textbf{Adjazenzmatrix} & $A(G)$ & $n \times n$ - Matrix \\
    \multicolumn{3}{@{}l}{\textit{mit Anzahl Kanten zwischen den Ecken}} \\
    \\
    \textbf{Inzidenzmatrix} & $B(G)$ & $n \times m$ - Matrix \\ 
    \multicolumn{3}{@{}l}{\textit{Ecke liegt auf Kante (0 oder 1)}} \\
    \\
    \textbf{Gradmatrix} & $D(G)$ & $n \times n$ - Diagonal-Matrix \\
    \multicolumn{3}{@{}l}{\textit{Grade der Knoten auf der Diagonalen}} \\
\end{tabular}

\subsection{Wege und Kreise}

Anzahl Wege der Länge $l$ von Knoten $i$ zu $j$ \\
Eintrag $(i,j)$ von $A(G)^l$ \textit{(Adjazenzmatrix hoch $l$)} \\

\begin{tabular}{@{}ll}
    Weg & Folge von Kanten \\
    Kreis & gleicher Anfangs- und Endpunkt \\
    einfacher Kreis & jede Kante höchstens einmal \\
    Eulerweg & jede Kante genau einmal \\
    Eulerkreis & jede Kante genau einmal \\
    Hamiltonweg & jeden Knoten genau einmal \\
    Hamiltonkreis & jeden Knoten genau einmal \\
\end{tabular} \\

\textit{Satz von Dirac} \\
ein Graph mit $n \geq 3$ Knoten mit Grad $\geq n/2$ hat einen Hamiltonkreis \\
\\
\textit{Satz von Ore} \\ 
ein Graph mit $n \geq 3$ mit ${deg}(v)+{deg(u) \geq n}$ für jedes Paar $u,v$ von nicht benachbarten Ecken hat einen Hamiltonkreis
