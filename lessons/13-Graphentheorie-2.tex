\subsection{Planare Graphen}
wenn er sich ohne Kantenkreuzungen zeichnen lässt

\subsection{Satz von Euler}
Für ein zusammenhängender, planarer Graph \\ 
mit $|V|$ Knoten, $|E|$ Kanten und $|R|$ Regionen gilt: \\
\\
$2 = |V| - |E| + |R|$

\subsection{Satz von Kuratovsky}
Ein Graph ist genau dann nicht planar, wenn er einen \\ 
Untergraphen vom Typ $K_{3,3}$ oder $K_5$ enthält

\subsection{Färbungen}
$ c : V \rightarrow C $ so dass $c(u) \neq c(v) $ falls $\{u,v\} \in E $ \\
\\
Abschätzung: $ 1 \leq \chi(G) \leq \Delta (G) + 1 $ \\
\\
\textit{Anzahl mögliche Färbungen mit $x$ Farben:}\\
\begin{tabular}{@{}ll}
    Graph mit $E = \emptyset$ & $P(G,x)=x^n$ \\
    Vollständiger Graph & $P(K_n,x)=x * \dots * (x-n+1)$ \\
    Baum & $P(T_n,x)=x * (x-1)^{n-1}$ \\
\end{tabular}

\subsection{Dekompositionsgleichung}
Graph $G = (V,E)$ mit Kante $e={a,b}$ \\

\begin{tabular}{@{}ll}
    $G-e$ & Graph $G$ unter Weglassung der Kante $e$ \\
    $G_e$ & Graph $G$ mit zusammengezogener Kante $e$ \\
    & unter Weglassung aller parallelen Kanten
\end{tabular} \\

\textit{Anzahl Färbungen von $G$ mit $x$ Farben:} \\
$P(G,x)=P(G-e,x)-P(G_e,x)$ \\

\textit{Ziel:} \\
Rückführung des Graphen auf Bäume und vollständige Graphen mit errechenbarer Anzahl von Färbungen \\

\textit{Chromatische Zahl eines Graphen:} \\
$\chi(G) = \min\{x \in \mathbb{N} : P(G,x) > 0\}$

\subsection{Gerüste / Spannbäume}
zusammenhängender, kreisfreier Unterbaum, \\
der alle Knoten aus $V$ enthält \\

\begin{tabular}{@{}ll}
    $G-e$ & Graph $G$ unter Weglassung der Kante $e$ \\
    $G/e$ & Graph $g$ unter Zusammenziehung der Kante $e$ \\ 
    & und Weglassen aller Schlingen
\end{tabular} \\

\textit{Anzahl der Gerüste des Graphen:} \\
$G$: $t(G)=t(G-e)+t(G/e)$ \\
\\
\textit{Ziel:} \\
Rückführung des Graphen $G$ auf Kreise und Bäume mit bekannter/errechenbarer Anzahl Gerüste

