\section{Zahlentheorie}

\subsection{Division mit Rest}
$ A = q * n + r$ wobei $ 0 \leq r < |n|$

\subsection{Kongruenz modulo n}
\begin{tabular}{lll}
    $ a \equiv b (\mod n)$ & $\iff$ & $n|(a - b)$ \\
    & $\iff$ & $\exists q : a-b = q * n$ \\
    & $\iff$ & $\exists q : a = b + q * n$ \\
\end{tabular}

\subsection{Euklidsche Algorithmus}
\begin{tabular}{rcrcrcr}
    963 & = & 4 & * & 218 & + & 91 \\
    218 & = & 2 & * & 91 & + & 36 \\
    91 & = & 2 & * & 36 & + & 19 \\
    36 & = & 1 & * & 19 & + & 17 \\
    19 & = & 1 & * & 17 & + & 2 \\
    17 & = & 8 & * & 2 & + & 1 \\
    8 & = & 2 & * & 1 & + & 0 \\
\end{tabular}

\subsection{Diophantischer Gleichung}
$n_1 * x + n_2 * y = n$

\subsection{erweiterter Euklidsche Algorithmus}
\begin{tabular}{ccccl}
    \textbf{67} & \textbf{-} & \textbf{1} & \textbf{0} & \\
    \textbf{24} & 2 * & \textbf{0} & \textbf{1} & \\
    19 * & 1 & 1 * & -2 * & \textit{19 = 67 \% 24} \\
    5 & 4 & -1 & 3 & \textit{2 = 67 div 24} \\
    4 & 1 & 4 & -11 & \textit{1 = 1 - 2 * 0} \\
    1 & & -5 & 14 & \textit{-2 = 0 - 2 * 1} \\
\end{tabular} \newline

\subsection{Chinesischer Restsatz}
$M_i = \frac{m}{m_i}$ \\
$M_i * y_1 \equiv 1 (\mod m_i)$ \\
$x = \sum_{i=1}^{k} r_i * M_i * y_i$

\subsection{Eulersche $\phi$-Funktion}
$\mathbb{Z}_n := \{ 0, 1, 2, \ldots, n-1\} $ \\
$\mathbb{Z}_n^* := \{ x \in \mathbb{Z}_n | x > 0 $ und $ ggT(x,n) = 1 \} $ \\
$|\mathbb{Z}_n^*| := $ Anzahl Elemente in $\mathbb{Z}_n^*$ \\

$\phi : \mathbb{N} \rightarrow \mathbb{N}, n \mapsto |\mathbb{T}_n^*| =: \phi (n) $ \\

\begin{tabular}{rcl}
    $\phi (p)$ & = & $p - 1$ \\
    $\phi (p * q)$ & = & $(p - 1) * (q - 1)$ \\
    $\phi (m)$ & = & $(p_1 - 1) * p_1^{r_1 - 1} * (p_2 - 1) * p_2^{r_2 - 1} * \ldots $ \\
\end{tabular}

\subsection{Primzahl}
$ n = p_1^{e_1} * p_2^{e_2} * p_3^{e_3} * \ldots *  p_n^{e_n} $

\subsection{kleiner Satz von Fermat}
$m^p \mod p = m \mod p$

\subsection{Primzahltest von Wilson}
falls $(n - 1)! + 1$ durch $n$ teilbar ist

